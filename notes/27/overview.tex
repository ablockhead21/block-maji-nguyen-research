\subsection{Technical Overview}\label{sec:overview}
In \figureref{paradigm}, we present the primary technical idea underlying the constructions of our correlation extractors,  which is also common to the protocols in \cite{ISIT:IMSW14,C:GIMS15}.

%%%
\begin{figure}[htp]%\footnotesize
\begin{boxedalgo}
\begin{enumerate}[leftmargin=5mm]
	\item Clients A and B have secret shares $r_A, r_B $ respectively from a correlation from the preprocessing step\\
	\hrule
	
	\item From an ensemble of pair of linear codes $\left\{(C_i,D_i)\colon i\text{ in the index set}\right\}$, Client A and B agree on a pair of linear codes $(C,D)$.
	
	\item Let \cE be a linear code that contains the linear span of all the coordinate-wise products of each codeword in $C$ with each codeword in $D$, represented by $\cE \supseteq C*D$. \\
	\hrule
	
	\item Client A picks a random codeword $(u_{-\gamma},\dotsc,u_{-1},u_1,\dotsc,u_\eta)\in C$.
	
	\item Client B picks a random codeword $(r_{-\gamma},\dotsc,r_{-1},r_1,\dotsc,r_\eta)\in D$.
	
	\item Client A picks a random codeword $(v_{-\gamma},\dotsc,v_{-1},v_1,\dotsc,v_\eta)\in\cE$. \\ 
	\hrule 
	
	\item Clients A and B use their secret shares to help client B securely compute $ z $, which can be used to resconstruct $ z_i = u_i r_i + v_i $ for $ i \in \{-\gamma, \dotsc, -1 \} $ using ereasure recovery algorithm for $ \cE $.
	\hrule 
	
	\item Client A outputs $(u_i,v_i)$ and Client B outputs $(r_i,z_i)$, for each $i\in\{-\gamma,\dotsc,-1\}$ 
\end{enumerate}
\end{boxedalgo}
\caption{Round-optimal correlation extractor design paradigm based on Massey secret sharing scheme.}
\label{fig:paradigm}
\end{figure}













%Note that only steps (1) and (7) require interaction. 
%Step (1), which is equivalent to secure coin tossing, can be performed by either party because the parties are semi-honest. 
%Step (7) is a two-round protocol (\cf~\cite{EC:WolWul06}). 
%Hence, the overall protocol is a two-round protocol, if the client who sends the first message for step (7) also performs step (1). 
%If \cE admits an efficient erasure recovery algorithm and $d(\cE) > \gamma$, then client B can efficiently perform step (8). 
%Note that, for every $i\in\{-\gamma,\dotsc,-1\}$, the outputs $(u_i,v_i)$ and $(r_i,z_i)$ are correct \ROLE[\bbF] secret shares. 
%The privacy of this protocol, which also subsumes the argument that the output \ROLE[\bbF] secret shares are independent, relies on the particulars of the leakage model and the choices of the ensemble of codes \cC, \cD, and \cE. \\

Following this paradigm, the security argument for the correlation extractor reduces to the observation that, given any particular non-trivial linear test, a random codeword from a random code fools this linear test. 
Note that linear leakage is a special type of leakage, and surprisingly protecting only against all linear leakages suffices to protect against arbitrary leakage. 
Thus, protection against linear leakage is not only necessary but also sufficient. 

More concretely, let  $\gamma, \eta \in \bbN$ and consider any non-zero $S=(S_{-\gamma},\dotsc, S_{-1}, S_1, \dotsc, S_\eta) \in \bbF^{\gamma + \eta}$. 
The linear function $L_S\colon\bbF^{\gamma + \eta}\to\bbF$ is defined by $L_S(x) = \sum_{i \in [-\gamma,\eta]} S_ix_i$ for $x \in \bbF^s$, where $[-\gamma, \eta] = \{-\gamma,\dotsc,-1,1,\dotsc,\eta\}$.
If $x$ is chosen uniformly at random from  $\bbF^{\gamma+\eta}$ then the output of $L_S(x)$ is uniformly random over \bbF, for all non-zero $S$. 
However, if $x$ is chosen uniformly at random from a vector subspace of $\bbF^{s}$, then $L_S$ is a constant function for every $S$ in the dual of this subspace.  
So, sampling $x$ from a {\em particular subspace} of $\bbF^{s}$ is unable to fool%
\footnote{
	A distribution $\cD$ over the sample space $\bbF^{\gamma+\eta}$ $\rho$-{\em fools} a linear test $L_S$ if the distribution of $L_S(x)$, where $x\drawn \cD$, is $\rho$-close to the uniform distribution over \bbF. 
}
every linear test.
Sampling $x$ from an {\em ensemble} of subspaces of $\bbF^{s}$, on the other hand, can fool every non-trivial linear test, except with an exponentially low probability. 
Such ensembles are called a {\em family of small-bias distributions.} 
Thererfore, the privacy of client A and client B, respectively, reduces to demonstrating that the ensemble of the codes $\{C_i\}$ and the ensemble of the codes $\{D_i\}$ satisfy this property.  
We emphasize that if there exists $S$ such that the linear test $L_S$ is, say, not fooled by the ensemble of codes $\{C_i\}$ then an adversarial client B can perform a 1-bit leakage to violate the privacy of client A. \\


\noindent\textbf{Instantiation for \cite{ISIT:IMSW14,C:GIMS15}.} 
These recent works on round-optimal correlation extractor construction create the ensemble of codes as follows. 
Let $\{C_i\}$ be the set of all linear codes, and $D_i$ is the dual code of $C_i$. 
Consequently, for any pair of linear codes $(C,D)$ in the ensemble, set $\cE$ as the linear code such that the sum of every element in the codeword is 0, referred to as the {\em parity code}. Since the parity code has distance 2 and it can efficiently recover one erasure, we have $\gamma=1$. 
%We emphasize that \cite{ISIT:IMSW14,C:GIMS15} use $\bbF=\GF{2}$, while \cite{C:BloMajNgu17} uses large fields with characteristic 2. 


%Moreover, intuitively, as the probability of fooling linear tests gets closer to 1, the protocol can have higher production (if permitted by the distance of \cE) and tolerate more leakage. \cmmt{I know that here you are referring to the expression in unpredictability lemma, but its hard to understand. We should rephrase this; I don't know how though.}

%In the terminology of leakage resilience,  
%\TODO{This approach does not give high production.} \dg{But, note that in above constructions, $d(\cE) = 2$ and hence, we can do only one erasure recovery. This implies that the above small-bias distributions cannot produce high production directly. \textsc{Add a footnote on how GIMS gets more than one OT.}} 

The approach mentioned above (where $D_k$ is the dual code of $C_k$) only achieves $d(\cE)=2$ and, consequently, $\gamma=1$. 
To achieve linear production, a potential approach is to ensure that $d(\cE)$ is linear.
%
%The discussion above reduces the goals of our work to the following. 
This approach reduces the goals of our work to the following. 
\begin{boxedalgo}
	\noindent{\bfseries Our Target Properties.} 
	Over suitably chosen finite fields
	\begin{enumerate}
		\item Construct the ensemble $\{(C_i,D_i)\}$ such that the linear code $\cE\supseteq C_i*D_i$ supports $\gamma$-erasure recovery for each $i$, and 
		\item The ensembles of codes $\{C_i\}$ and $\{D_i\}$ are both families of small bias distributions. 
	\end{enumerate}
\end{boxedalgo} 

\noindent\textbf{Related work of \cite{TCC:MeiPrzWul07,ICALP:PrzWul08} using secret-sharing schemes of \cite{C:CheCra06}.}%
\footnote{
	We emphasize that the original constructions of \cite{TCC:MeiPrzWul07,ICALP:PrzWul08} used Reed-Solomon codes over large fields. 
	We believe that near-MDS AG codes (\ie, the Singleton Bound $d\leq n-k+1$ is satisfied with a small multiplicative slack) over constant size fields can also be used in their protocols.  
} 
Using Algebraic Geometry (AG) Codes, there are near-MDS codes%
\footnote{
	A maximum distance separable code, or MDS code, satisfies the Singleton Bound tightly $d= n-k+1$. 
	A near-MDS code roughly achieves this bound with a slight multiplicative slack. 
}
with one of the properties mentioned above, namely, target property (1). 
%There are Algebraic Geometry Code, which are near-MDS, with target property (1). 
There is an appropriate AG Code $C^*$ such that, if $C=D=C^*$ then the code \cE has linear distance. 
We have already discussed that any single code cannot fool all linear tests, so this idea clearly {\em does not} suffice for our goals. 
However, \cite{TCC:MeiPrzWul07,ICALP:PrzWul08} use this idea to achieve the partial goal of protection against the leakage of individual bits of the secret shares. 
In fact, the construction is robust to linear tests $S$ with small support, and there are 1-bit linear leakages with a large support that render this construction insecure. 
To achieve our goals, {we need to fool every linear test}, including the ones with large support.  %\dg{It does not fool large linear tests.}
%\TODO{Generalization from probing to arbitrary leakage.} 
\\

\noindent\textbf{Related work of \cite{FOCS:IKOS09}.} 
The approach of Ishai~\etal~\cite{FOCS:IKOS09}, which also uses this AG Code $C^*$, can be interpreted in the paradigm presented in \figureref{paradigm}, albeit a minor modification. 
The ensemble of codes $\{C_i\}$ is {\em non-linear} code,% 
\footnote{
	Concatenation of an AG code with a binary representation of the field elements, a majority encoding of each bit, and a randomized encoding for a suitable constant-size functionality. 
	The ensemble is constructed by enumerating all outcomes of the probabilistic encoding procedures.  
}
that is a family of small biased distribution, while the code $D$ is an appropriate fixed linear code. 
%The code \cE is not defined by a coordinate-wise product of the each pair of codewords in \cC and \cD, but by coordinate-wise applying a suitable (constant-size) functionality. 
%Moreover, most crucially, their approach only achieves security against one client. 
This approach, therefore, achieves security for client A against an adversarial client B. 
Recall that this is the primary reason that their construction is not round optimal. 
To achieve our goals, {we need to construct $\{(C_i, D_i)\}$ such that both $\{C_i\}$ and $\{D_i\}$ are families of small bias distributions.}  