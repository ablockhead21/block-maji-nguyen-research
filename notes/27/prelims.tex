\section{Preliminaries}\label{sec:prelims}
\textcolor{red}{As a note, we shall keep these preliminaries general so that they apply to each section below. Any additional background will be presented on a per section basis.}

We denote random variables and distributions by capital letters, for example $X$, and the values taken by small letters, for example $X = x$ or $x \drawn X$.
For a positive integer $n$, we write $[n]$ and $[-n]$ to denote the sets $\{ 1,\dotsc, n \}$ and $\{ -n,\dotsc, -1 \}$, respectively.

\subsection{Functionalities and Correlations}\label{sec:prelim-func-corr}
We define some useful functionalities and correlations.

\paragraph{Oblivious Transfer.}
Oblivious transfer, denoted by \OT, is a two-party functionality which takes $(m_0,m_1) \in \zo^2$ as input from Alice and $c \in \zo$ as input from Bob and outputs $m_c$ to Bob.

\paragraph{Oblivious Linear-function Evaluation.}
For a field $\bbF$, oblivious linear-function evaluation over $\bbF$, denoted by \OLE[\bbF], is a two-party functionality which takes $(a,b) \in \bbF^2$ as input from Alice and $x \in \bbF$ as input from Bob, and outputs $z = ax+b$ to Bob.
In particular, \OLE[\bbF] is a natural extension of \OT to any finite field, and \OLE[\GF{2}] is functionally equivalent to \OT by the following observation: $m_b = (m_1 - m_0)b + m_0$.

\paragraph{Random Oblivious Transfer Correlation.}
The random oblivious transfer correlation, denoted by \ROT, is a correlation that samples $m_0,m_1,c \getsr \zo$ uniformly and independently at random and provides secret shares $r_A = (m_0,m_1)$ and $r_B = (c,m_c)$ to Alice and Bob, respectively.

\paragraph{Random Oblivious Linear-function Evaluation Correlation.}
For a field $\bbF$, the random oblivious-linear function evaluation correlation over $\bbF$, denoted by \ROLE[\bbF], is a correlation which that samples $a,b,x \getsr \bbF$ uniformly and independently at random, and provides secret shares $r_A = (a,b)$ and $r_B = (x, z \defeq ax+b)$ to Alice and Bob, respectively.
In particular, we let \ROLE denote the \ROLE[\GF{2}] correlation, and note that \ROT and \ROLE are functionally equivalent correlations.

\paragraph{Inner-product Correlation.}
For a field $\bbF$ and $n \in \bbN$, the inner-product correlation over $\bbF$ of size $n$, denoted by \IP[\bbF^n], is a correlation that samples random $r_A = (x_0,\dotsc, x_{n-1}) \in \bbF^n$ and $r_B = (y_0,\dotsc, y_{n-1}) \in \bbF^n$ subject to the constraint that $x_0 + y_0 = \sum_{i=1}^{n-1} x_iy_i$.
The secret shares of Alice and Bob are $r_A$ and $r_B$, respectively.

For $m \in \bbN$, the functionality or correlation $\cF^m$ represents the functionality that implements $m$ independent copies of any functionality or correlation $\cF$.

\subsection{Correlation Extractors}\label{sec:prelim-corr-ext}

\textcolor{red}{Note this definition may be moved to section 1.1.}
\begin{definition}[$(n,m,t,\eps)$-Correlation Extractor]
	Let $(R_A, R_B)$ be a correlated private randomness such that the size of the secret share of each party is $n$-bits.
	An $(n,m,t,\eps)$-correlation extractor for $(R_A, R_B)$ is a two-party interactive protocol in the $\leaky{(R_A,R_B)}{t}$-hybrid that securely implements the $\ROT[m/2]$ functionality against information-theoretic semi-honest adversaries with $\eps$ simulation error.
\end{definition}

\paragraph{Leakage Model.}

\paragraph{Correctness.}

\paragraph{Privacy.}

\subsection{Fourier Analysis over Finite Fields}\label{sec:prelim-fourier}

\subsection{Distributions and Min-Entropy}\label{sec:prelim-min-ent}

\subsubsection{Family of Small-Bias Distributions}\label{sec:prelim-small-bias}

\subsubsection{Distributions over Linear Codes}\label{sec:prelim-codes}





