\section{Survey}

The bulk of this survey comes from \cite{FurSim2013}, which provides a comprehensive history of various extremal graph problems up to mid 2013. Any newer result found in relation to extremal graph problems related to the Zarankiewicz problem mainly concern new upper bounds or slight variations on the problem.

\subsection{Preliminaries and Notations}

\subsection{Best Bounds}
\begin{theorem}[F\"uredi~\cite{Furedi94,Furedi96-2}]
	If $q \neq 1, 7, 9, 11, 13$ and $n = q^2 + q + 1$, then
	\begin{align*}
		\mathbf{ex}(n,C_4) \leq \dfrac{1}{2}q(q+1)^2
	\end{align*}
	Moreover, if $q$ is a power of a prime, then
	\begin{align*}
		\mathbf{ex}(n,C_4) = \dfrac{1}{2}q(q+1)^2.
	\end{align*}
\end{theorem}

A slight change in this bound from a more specific $q$ is due to Firke, Kosek, Nash, and Williford.
\begin{theorem}[\cite{FKNW12}]
	Suppose $q$ is even and $q > q_0$. Then,
	\begin{align*}
		\mathbf{ex}(q^2+q, C_4) \leq \dfrac{1}{2}q(q+1)^2 - q.
	\end{align*}
	Consequently, if $q > q_0$, $q = 2^k$, and $n = q^2 + q$, then
	\begin{align*}
		\mathbf{ex}(n,C_4) = q(q+1)^2 - q.
	\end{align*}
\end{theorem}