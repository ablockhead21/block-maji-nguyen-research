\section{Malicious Combiner}

\begin{figure}[!hp]
\begin{boxedalgo}
{\bfseries Private Input.} Alice has private input $(m_0,m_1)$ and Bob has private input $b$.\\

{\bfseries Common Randomness.} Both parties use CRS to sample generator matrix $G \in \zo^{n/2 \times (n+1)}$ of the form $[I_{n/2\times n/2}\Vert P]$ where $P$ is a random Toeplitz matrix.
Let $C$ be the code spanned by $G$, and $C^\perp$ be the dual code spanned by $H$.\\

{\bfseries Protocol.}
\begin{enumerate}
	\item Alice secret shares $(m_0, m_1)$ into $\{s_{i,j}\}_{i \in [n],j \in \zo}$ as follows.
	Alice samples $c_u = (m_0,u_1,\dotsc, u_n) \getsr C$ and $c_v = (v, v_1, \dotsc, v_n) \getsr C_\parity$ such that $m_1 = m_0 \xor v$.
	For $ i\in [n]$, set $s_{i,0} = v_i$ and $s_{i,1} = u_i \xor v_i$.
	Send $(s_{i,0}, s_{i,1})$ to server $S_i$ for all $i \in[n]$.
	
	\item Bob samples random $(b, x_1,\dotsc, x_n) \getsr C^\perp$, and sends $x_i$ to server $S_i$ for all $i \in [n]$, and receives output $y_i$.
	
	\item Bob computes $y = \xor_{i \in [n]} y_i$.
\end{enumerate}
	
\end{boxedalgo}
\label{fig:combiner}
\caption{Malicious secure OT combiner.}
\end{figure}

\begin{theorem}\label{thm:main}
	The protocol in \figureref{combiner} is an $(n,t)$ malicious secure OT combiner for $t < n/2 - \delta$, for all $\delta \geq 1$.
\end{theorem}

To prove the theorem we shall use the following lemma.
\begin{lemma}\label{lem:secret-info}
	Fix a set $\cB \subseteq [n]$ of corrupt servers such that $|\cB| < n/2 - \delta$ and let $\overline{\cB} = [n]\setminus \cB$.
	Over the choice of generator matrix $G$ for codespace $C$, with probability at least $1-2^{-\delta}$ the following holds.
	For any $(x_1,\dotsc, x_n) \in \zo$, define
	\begin{align*}
		\cJ = \{ (i,x_i)\colon i \in \overline{\cB} \} \union \{ (i,j) \colon i \in\cB, j \in \zo \}.
	\end{align*}
	Then,
	\begin{enumerate}
		\item If the set $\{x_i \colon i \in \overline{\cB} \}$ does not extend to a valid codeword in $C^\perp$, then the shares $\{ s_{i,j} \}_{(i,j) \in \cJ}$ give no information about $(m_0,m_1)$.
		\item If the set $\{x_i\colon i \in \overline{\cB}\}$ extends to a valid codeword in $C^\perp$ starting with $b \in \zo$, then the shares $\{s_{i,j}\}_{(i,j)\in \cJ}$ can be used to reconstruct $m_b$ and give no information about $m_{1-b}$.
	\end{enumerate}
\end{lemma}
\begin{proof}
	For any fixed set $\cB\subseteq [n]$ of corrupt servers where $|\cB| < n/2 - \delta$, by \lemmaref{qual} we have that the probability $\cB$ is qualified for $G_0$ is at most $2^{-n/2 + |\cB|} = 2^{-\delta}$.
	Then, we consider the case where $\cB$ is not qualified for $G_0$.
	By \lemmaref{min-qual-secret}, since $\cB$ is not qualified for $G_0$, the set $(\cB, 0) \union (\cB,1) \subseteq \cJ$ is not qualified for reconstructing $m_0$.
	
	Next, if the vector $x_{\overline{\cB}}$ cannot be extended to a valid codeword in $C^\perp$, by \lemmaref{min-qual-secret} we have that $\cJ$ cannot be qualified for $m_1$ or $v^* = m_0 \xor m_1$.
	Therefore in this case the shares $\{ s_{i,j} \}_{(i,j) \in \cJ}$ reveal no information about $m_0$ and $m_1$.
	
	If $x_{\overline{\cB}}$ extends to a valid codeword in $C^\perp$, this implies that $\overline{\cB}$ is qualified for $H_0$ \textcolor{red}{except with probability...}.
	Since $\overline{\cB}$ is qualified for $H_0$, it cannot be the case that the codeword found by extending $x_{\overline{\cB}}$ starts with both $0$ and $1$.
	Therefore if $b$ is the starting bit of this codeword, by \lemmaref{min-qual-secret} we have that $\cJ$ is qualified for only $m_b$ and not $m_{1-b}$.\qed
\end{proof}

\begin{proof}[of \theoremref{main}]
	First we consider security against malicious Bob corrupting a set of $\cB \subseteq[n]$ servers such that $|\cB| = t$.
	Define $\overline{\cB} = [n]\setminus\cB$.
	We construct $\pred{Sim}_B$ as follows:
	\begin{enumerate}
		\item $\pred{Sim}_B$ receives $(x_1',\dotsc, x_n') \in \zo$ from Bob.
		\item For $i \in \overline{\cB}$, extend $x_i'$ to a valid codeword in $C^\perp$.
		If reconstruction succeeds, let $x \in \zo$ be the first coordinate.
		Else, let $x = \bot$.
		\item If $x \neq \bot$, $\pred{Sim}_B$ submits $x$ to $\cF_{OT}$ and receives $m_x$.
		$\pred{Sim}_B$ the samples $m_{1-x} \getsr \zo$.
		If $x = \bot$, $\pred{Sim}_B$ samples $m_0,m_1 \getsr \zo$.
		\item $\pred{Sim}_B$ generates sharing $\{ s_{i,j} \}_{i \in [n], j \in \zo}$ of $(m_0, m_1)$.
		\item For each $i \in \overline{\cB}$, $\pred{Sim}_B$ sends $s_{i, x_i'}$ to Bob.
		For each $i \in \cB$, $\pred{Sim}_B$ sends $(s_{i,0}, s_{i,1})$ to Bob.
	\end{enumerate}
	We now use \lemmaref{secret-info} to argue the view of malicious Bob and $\pred{Sim}_B$ are indistinguishable with high probability.
	Note that for corrupted servers $\cB \subseteq [n]$ where $|\cB| < n/2 - \delta$, the probability that $\cB$ is qualified for $G_0$ is at most $2^{-\delta}$.
	This implies that a malicious Bob in the real world, upon seeing shares $\{ (s_{i,0}, s_{i,1}) \}$ for $i \in \cB$, can reconstruct the secret $m_0$ (by \lemmaref{min-qual-secret}).
	Then Bob can pick dual codeword $(1, x'_1,\dotsc,x'_n) \in C^\perp$ and submit $x\sub{n}$, receiving $s_{i, x'_i}$ for all $i \in \overline{\cB}$.
	Then by \lemmaref{min-qual-secret}, Bob can reconstruct $m_1$ as well.
	Thus with probability $2^{-\delta}$ the two views are distinguishable.
	
	Otherwise, with probability at least $1 - 2^{-\delta}$, the set of corrupted servers $\cB$ is unqualified for $G_0$.
	Then we consider two cases.
	Let $x'\sub{n} \in \zo^n$ be the submitted value by Bob.
	In the first case, suppose $\{ x_i' \colon i \in \overline{\cB} \}$ cannot be extended to a valid codeword in $C^\perp$ \textcolor{red}{(this happens with some small probability...)}.
	By \lemmaref{secret-info}, this implies the shares Bob receives in the real world give no information about $(m_0,m_1)$, and thus Bob must guess these values.
	In this same case, if $\pred{Sim}_B$ is unable to extend $\{ x_i' \colon i \in \overline{\cB} \}$ to a codeword in $C^\perp$, then the simulator chooses $m_0, m_1 \getsr \zo$.
	Then again by \lemmaref{secret-info}, the secret sharing of $(m_0, m_1)$ that $\pred{Sim}_B$ submits to Bob gives no information about $(m_0,m_1)$.
	Thus the views are indistinguishable in this case.
	
	In the other case suppose $\{x_i \colon i \in \overline{\cB} \}$ does extend to a valid codeword in $C^\perp$ \textcolor{red}{(happens with some large probability...)}.
	Let $x \in \zo$ be the first bit of this codeword.
	By \lemmaref{secret-info}, this implies the shares Bob receives in the real world reconstruct to value $m_x$, and give no information about $m_{1-x}$.
	So Bob must guess the value of $m_{1-x}$.
	In the same case, $\pred{Sim}_B$ extends $\{ x_i' \colon i \in \overline{\cB} \}$ to a codeword in $C^\perp$ and gets the first bit of this codeword $x$.
	The simulator then submits $x$ to $\cF_{OT}$ and receives $m_x$, then picks $m_{1-x} \getsr \zo$.
	The simulator then generates a sharing of $(m_0, m_1)$ and submits this sharing to Bob.
	By \lemmaref{secret-info}, the values submitted by the simulator exactly give Bob $m_x$ and reveal no information about $m_{1-x}$.
	Thus the views are indistinguishable in this case.
	
	Now we consider a malicious Alice corrupting a set of servers $\cA$ such that $|\cA| = t$.
	Let $\overline{\cA} = [n]\setminus\cA$.
	We construct $\pred{Sim}_A$ as follows:
	\begin{enumerate}
		\item $\pred{Sim}_A$ samples $(0, x_1,\dotsc, x_n) \getsr C^\perp$.
		\item For $i \in \cA$, $\pred{Sim}_A$ sends $x_i$ to Alice (\textcolor{red}{or Environment}).
		\item $\pred{Sim}_A$ receives $s_{i,x_i}'$ for $i \in \cA$ and $(s_{i,0}', s_{i,1}')$ for $i \in \overline{\cA}$ from Alice.
		\item $\pred{Sim}_A$ samples another $(1, x_1', \dotsc, x_n') \getsr C^\perp$ such that $x_\cA = x'_\cA$.
		\item $\pred{Sim}_A$ reconstructs 
	\end{enumerate}
\end{proof}





