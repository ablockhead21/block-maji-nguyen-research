\section{Better Upper Bound}
Let $ S, T $ be ordered sequences with $S = \{s_1, s_2, \cdots, s_m \}  $ and $ T = \{t_1, t_2, \cdots, t_m\} $. Suppose $ m = \delta N $.  Let $ B = \{(s_i,t_i): i \in [m] \} $.\\
We shall need a function counting the number of solutions $ s_i +t_i = s_j + t_k $ for our combinatorial problem. Given $ f : \bbZ_N \times \bbZ_N \rightarrow \bbC $, and $ g, h : \bbZ_N \rightarrow \bbC $, we define
$$ \Lambda(f,g,h) = \frac{1}{N^3} \sum\limits_{x,y,z \in \bbZ_N} f(x,y) g(z) h(x+y-z) .$$

\begin{lemma} \label{lem:to-Fourier}
	Given $ f : \bbZ_N \times \bbZ_N \rightarrow \bbC $, and $ g, h : \bbZ_N \rightarrow \bbC $, we have
	$$ \Lambda(f,g,h) = \sum\limits_{k \in \bbZ_N} \widehat{f}(k,k) \widehat{g}(-k) \widehat{h}(-k) .$$
\end{lemma}
\begin{proof}
	Directly by Fourier transformation, we have:
	\begin{align*}
		\Lambda(f,g,h) &= \frac{1}{N^3} \sum\limits_{x,y,z\in \bbZ_N} f(x,y)g(z)h(x+y-z)\\
		&=\frac{1}{N^3} \sum\limits_{x,y,z\in \bbZ_N}\left(\sum\limits_{a,b\in\bbZ_N} \widehat{f}(a,b) e\left(\frac{ax + by}{N}\right) \right)\left(\sum\limits_{c\in \bbZ_N}\widehat{g}(c) e\left(\frac{cz}{N}\right) \right)\left(\sum\limits_{d\in \bbZ_N}\widehat{h}(d)e\left(\frac{d(x+y-z)}{N}\right) \right)\\
		&=\frac{1}{N^3}\sum\limits_{a,b,c,d}\widehat{f}(a,b) \widehat{g}(c) \widehat{h}(d) \sum\limits_{x,y,z}e\left(\dfrac{(a+d)x + (b+d)y + (c-d)z}{N}\right)\\
		&= \frac{1}{N^3}\sum\limits_{a\in\bbZ_N} \widehat{f}(a,a) \widehat{g}(-a) \widehat{h}(-a)\sum\limits_{x,y,z}e\left(0\right)\\
		&= \sum\limits_{a\in\bbZ_N} \widehat{f}(a,a) \widehat{g}(-a) \widehat{h}(-a) & \qedhere
	\end{align*}
\end{proof}

\noindent Note that if $ \mathbbm{1}_{B}, \mathbbm{1}_S, \mathbbm{1}_T $ are the indicator functions of the sets $ B, S, T $, respectively, then  $ N^3 \Lambda(\mathbbm{1}_B, \mathbbm{1}_S, \mathbbm{1}_T) $ counts the number of solutions in $ B $. For simplicity, from now let $ f = \mathbbm{1}_B, g = \mathbbm{1}_S, h = \mathbbm{1}_T $. 

\subsection{Few solutions to our combinatorial problem implies big Fourier coefficient.} The following lemma states that a lack of solutions to our combinatorial problem in $ B $ leads to the existence of a large Fourier coefficient.  
\begin{lemma} \label{lem:big-Fourier}
	One of the following must hold:
	\begin{enumerate}
		\item $ \Lambda(f,g,h) > \dfrac{\delta^3}{2N}$
		\item There exists a non-zero $ k $ such that $ \abs{\hat{f}(k,k)} \geq \dfrac{\delta^2}{2N} $
	\end{enumerate}
\end{lemma}

\begin{proof}
	Suppose  $ \Lambda(f,g,h) \leq \dfrac{\delta^3}{2N}$, we have:
	\begin{align*}
		\dfrac{\delta^3}{2N} & \geq \Lambda(f,g,h) \\
		& = \sum\limits_{k \in \bbZ_N} \widehat{f}(k,k) \widehat{g}(-k) \widehat{h}(-k)  & \text{by \lemmaref{to-Fourier}} \\
		& = \widehat{f}(0,0) \widehat{g}(0) \widehat{h}(0) + \sum\limits_{k \neq 0} \widehat{f}(k,k) \widehat{g}(-k) \widehat{h}(-k) \\
		& = \dfrac{\delta^3}{N} + \sum\limits_{k \neq 0} \widehat{f}(k,k) \widehat{g}(-k) \widehat{h}(-k) \\
	\end{align*}
	Therefore, 
	\begin{align*}
		\dfrac{\delta^3}{2N} 
		& \leq \abs{\sum\limits_{k \neq 0} \widehat{f}(k,k) \widehat{g}(-k) \widehat{h}(-k)} \\
		& \leq \sup_{k \neq 0} \abs{\widehat{f}(k, k)}\abs{\sum\limits_{k \in \bbZ_N}  \widehat{g}(-k) \widehat{h}(-k)} \\
		& = \sup_{k \neq 0} \abs{\widehat{f}(k, k)} N \abs{\langle \widehat{g}, \widehat{h} \rangle} \\
		& = \sup_{k \neq 0} \abs{\widehat{f}(k, k)} \abs{\langle f, g \rangle}  & \text{by \lemmaref{}} \\
		& \leq \sup_{k \neq 0} \abs{\widehat{f}(k, k)} \| g \|_{L^2} \| h \|_{L^2} \\
		& \leq \sup_{k \neq 0} \abs{\widehat{f}(k, k)} \cdot \sqrt{\delta} \cdot \sqrt{\delta}  		
	\end{align*} 
Hence, there exists a non-zero $ k $ such that $ \abs{\widehat{f}(k, k)} \geq \dfrac{\delta^2}{2N} $.
\end{proof}

\noindent Note that the first case in \lemmaref{big-Fourier} implies that there exists a non-trivial solution to our combinatorial problem since $ N^3 \Lambda(f,g,h) \geq \dfrac{\delta^3 N^2}{2} \geq \delta N $ when $ N $ is large enough, namely, $ N \geq \dfrac{2}{\delta^2} $.

\subsection{Big Fourier coefficient implies density increment}
\begin{lemma} \label{lem:e-ineq}
	For any real number $ x $, we have $ \abs{e(x) - 1} \leq 2 \pi \| x \|_{\bbR_\bbZ}  $.
\end{lemma}

\begin{lemma} \label{lem:density-increment}
	Let $ f : \bbZ \times \bbZ_N \rightarrow \bbC $ be a function satisfying $ \sum\limits_{n \in \bbZ \times \bbZ_N} f(n) = 0 $ and $\sum\limits_{n \in \bbZ \times \bbZ_N} \abs{f(n)} \leq \delta N $. Suppose that there exists $ k $ such that $ \abs{\widehat{f}(k, k)} \geq \dfrac{\delta^2}{2N} $. Then there exist two arithmetic progressions $ P_1 $ and $ P_2 $ of length at least $ \dfrac{\delta \sqrt{N}}{4\pi} $ such that $ f $ has mean value at least $ \dfrac{\delta^2}{8N} $ on $ P_1 \times P_2 $.
\end{lemma}

\begin{proof}
	Note that $ \widehat{f}(0,0) = 0 $, so $ k \neq 0 $. By Dirichlet's approximation theorem there exist coprime integers $ b $ and $ h $ such that $ 1 \leq b \leq h \leq \sqrt{N} $ and that $ \abs{\dfrac{k}{N} - \dfrac{b}{h}} \leq \dfrac{1}{h \sqrt{N}} $. Thus $ \left\| \dfrac{k}{h}\right\|_{\bbR_\bbZ/\bbZ} \leq \dfrac{1}{\sqrt{N}} $. \\
	We divide the set $ \{0, 1, \cdots, N-1\} $ into $ h $ congruence classes modulo $ h $ as following. 
	$$ \cC(a) = \{a , a+h, a+ 2h, \cdots\} \cap \{0, 1, \cdots, N-1\} = \{a, a+h, \cdots, a+(N_a -1 )h \} $$
	Then we divide $ [0, N_a -1] $ into $ M $, which is chosen sometime later, intervals of same length $ M' = N_a/M $ and define
	$$ J_m(a) = \{a + j_m h , a+ (j_m +1)h, \cdots, a + (j_m + M' -1)h \} $$
	Therefore, we have 
	$$ \bbZ_N \times \bbZ_N = (\cup_{a \in [h]} \cC(a)) \times (\cup_{a \in [h]} \cC(a)) = \cup_{a \in [h]} \cup_{a' \in [h]}' \cup_{m =1}^{M} \cup_{m' = 1}^M J_m(a) \times J_{m'}(a'). $$
	Furthermore, we have 
	\begin{align*}
		\dfrac{\delta^2}{2N} &\leq \abs{\widehat{f}(k,k)} = \dfrac{1}{N^2} \abs{\sum\limits_{(x,y) \in \bbZ_N \times \bbZ_N}  f(x,y) \cdot e\left(-\dfrac{(x+y)k}{N}\right)}\\
		& = \dfrac{1}{N^2} \abs{\sum\limits_{a = 1}^h \sum\limits_{a' = 1} ^h \sum\limits_{m = 1}^M \sum\limits_{m' = 1}^M \sum\limits_{(x,y) \in J_m(a) \times J_{m'}(a')} f(x,y) e\left(-\dfrac{(x+y)k}{N}\right)}
	\end{align*}
	Each class $ \cC(a) $ has size at least $ \dfrac{N}{h} -1 $ and each $ J_m(a) $ has size $ L = \floor{\dfrac{N_a}{M}} \in \left[\dfrac{N}{hM} - 2, \dfrac{N}{hM} \right] $. \\
	Let $ \dfrac{k}{N} = \dfrac{b}{h} + \theta $ with $ \abs{\theta} \leq \dfrac{1}{h\sqrt{N}} $. For $ x \in J_m(a) $, let $ x = a + h(j_m + l) $, and for $ y \in J_{m'}(a') $, let $ y = a' + (j_{m'} + l')h $. Then we have
	\begin{align*}
	\abs{e\left(-\dfrac{(x+y)k}{N}\right)} 
	&= \abs{e\left(- \left( \dfrac{b}{h} + \theta \right) (a + a' + h(j_m + j_{m'} + l + l'))\right)} 	 \\
	& = \abs{e\left( -\dfrac{(a+a')b}{h}\right) e\left( (a + a' + h(j_m + j_{m'} + l + l')) \theta\right)} \\
	& \leq \abs{ e\left( (a + a' + h(j_m + j_{m'} + l + l')) \theta\right)}\\
	& \leq \abs{e(a+ a' + h(j_m +j_{m'})) + e(a+ a' + h(j_m +j_{m'})) (e(h(l+l')\theta) - 1)} \\
	& \leq 1 + \abs{e(h(l+l')\theta) - 1}  \\
	& \leq 1 + 2\pi  \| h(l+l')\theta \|_{\bbR /\bbZ} \\
	& \leq 1 + \dfrac{2 \pi  (L+L)} {\sqrt{N}} \\
	& = 1 + \dfrac{4 \pi L}{\sqrt{N}} 
	\end{align*}
	Hence
	\begin{align*}
	\dfrac{\delta^2}{2N} 
	&\leq \dfrac{1}{N^2}\abs{\sum\limits_{a = 1}^h \sum\limits_{a' = 1} ^h \sum\limits_{m = 1}^M \sum\limits_{m' = 1}^M \sum\limits_{(x,y) \in J_m(a) \times J_{m'}(a')} f(x,y)} + \dfrac{1}{N^2} \sum\limits_{(x,y) \in \bbZ \times \bbZ} \abs{f(x,y)} \dfrac{4 \pi L}{\sqrt{N}} \\
	& \leq  \dfrac{1}{N^2} \sum\limits_{a = 1}^h \sum\limits_{a' = 1} ^h \sum\limits_{m = 1}^M \sum\limits_{m' = 1}^M \abs{\sum\limits_{(x,y) \in J_m(a) \times J_{m'}(a')} f(x,y)} + \dfrac{4 \pi L \delta}{N\sqrt{N}}	
	\end{align*}
	Now, we choose $ M $ large enough so that $ \dfrac{4 \pi L \delta}{N\sqrt{N}} \approx \dfrac{\delta^2}{4N} $. Then we get
	$$ \dfrac{1}{N^2}\sum\limits_{a = 1}^h \sum\limits_{a' = 1} ^h \sum\limits_{m = 1}^M \sum\limits_{m' = 1}^M \abs{\sum\limits_{(x,y) \in J_m(a) \times J_{m'}(a')} f(x,y)} \geq \dfrac{\delta^2}{4N} $$
	Since $ \sum\limits_{n \in \bbZ \times \bbZ_N} f(n) = 0  $, 
	$$ \dfrac{1}{N^2}\sum\limits_{a = 1}^h \sum\limits_{a' = 1} ^h \sum\limits_{m = 1}^M \sum\limits_{m' = 1}^M \max \left(0, \sum\limits_{(x,y) \in J_m(a) \times J_{m'}(a')} f(x,y) \right) \geq \dfrac{\delta^2}{8N} $$
\end{proof}
	
	\noindent By averaging argument, there exist some arithmetic progressions $ J_m(a) $ and $ J_{m'}(a') $ of length at least $ \dfrac{\delta \sqrt{N}}{4\pi} $ such that 
	$$ \dfrac{1}{\abs{J_m(a)} \abs{J_{m'}(a')}} \sum\limits_{(x,y) \in J_m(a) \times J_{m'}(a)} f(x,y) \geq \dfrac{\delta^2}{8N} .$$

\subsection{Iteration.}
Let $ g = \mathbbm{1}_B - \dfrac{\delta}{N} \mathbbm{1}_{\bbZ_N \times \bbZ_N} $. Then according to \lemmaref{big-Fourier} and the assumption that $ B $ does not contain any non-trivial solutions, we must be in the second case, namely, there exists $ k \neq 0 $ such that $ \widehat{\mathbbm{1}_B}(k,k) \geq \dfrac{\delta^2}{2N} $. So $ \widehat{g}(k, k) \geq \dfrac{\delta^2}{2N} $. By applying \lemmaref{density-increment}, there exist two arithmetic progressions $ P_1  = \{ a + h(j_m + l): l \in \bbZ_{N_1} $ and $ P_2  = \{ a' + h(j_{m'} + l): l \in  \bbZ_{N_1} \} $ such that the mean value of $ g $ on $ P_1 \times P_2 $ is at least $ \dfrac{\delta^2}{8N} $, where $ N_1 = \abs{P_1} \geq \dfrac{\delta \sqrt{N}}{4 \pi} $. Let $ B_1 = \{ (l, l') \in \bbZ_{N_1} \times \bbZ_{N_1}: (a + h(j_m + l), a' + h(j_{m'} + l')) \in (P_1 \times P_2) \cap A \} $. Thus, we obtain a subset $ B_1 $ of $ \bbZ_{N_1} \times \bbZ_{N_1} $ with 
$$ \abs{B_1} \geq \left(\dfrac{\delta}{N} + \dfrac{\delta^2}{8N}\right) N_1 \geq \dfrac{\delta + \delta^2/8} {N_1} = \dfrac{\delta_1}{N_1} $$

