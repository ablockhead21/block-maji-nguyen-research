\section{Our Combinatorial Problem}
\begin{frame}{Our Combinatorial Problem}

In our CRYPTO 2017 Paper \cite{C:BloMajNgu17}, we introduced the following problem

{\setbeamercolor{block title}{bg=ForestGreen, fg=white}
\begin{block}{Our Problem}
	Find two ordered sets $S = \left\lbrace s_1,\dotsc, s_m\right\rbrace$ and $T = \left\lbrace t_1,\dotsc, t_m\right\rbrace$ such that
	\begin{itemize}
		\item $s_i$ and $t_i$ are non-negative integers for every $i$
		\item $s_i + t_j < n$ for every $i,j$
		\item $s_i + t_j \neq s_k + t_k$ for every $i,j,k$ that are not all identical
		\item $m$ is maximized
	\end{itemize}
\end{block}}

\onslide<2->{We are interested in upper bounding $m$}

\onslide<3->{
{\setbeamercolor{block title}{bg=Red, fg=white}
\begin{block}{Question}
	For large enough $n$, is $m = O(n)$ possible?
\end{block}}
}
\begin{itemize}
	\item<4-> $m = O(n)$ is the best bound we can hope for
\end{itemize}

\end{frame}

\begin{frame}{Our Results}
	We prove the following result
	{\setbeamercolor{block title}{bg=Coral, fg=white}
	\begin{theorem}[Our Problem is Sublinear]
		For our combinatorial problem, $m = O\left(\dfrac{n}{\log^*n}\right)$.
	\end{theorem}}

\begin{itemize}
	\item<2-> Here $\log^*$ is defined with respect to base 2 as
	\begin{align*}
		\log^* n = \begin{cases}
			0 & n\leq 1\\
			1 + \log^*(\log n) & n > 1
		\end{cases}
	\end{align*}
	\item<3-> To gain insight on how to prove this result, we examine similar problems and the techniques used to prove them
\end{itemize}

\end{frame}

\begin{frame}{Connection: 3-Free Sets}
Our combinatorial problem is a generalization of the well-know \textit{3-free set problem}

\begin{block}{3-Free Set Problem}
	Find $A \subseteq \{1,\dotsc,n\}$ such that
	\begin{itemize}
		\item $A$ does not contain any 3-term arithmetic progression
		\item $|A|$ is maximized
	\end{itemize}
\end{block}
\begin{itemize}
	\item<2-> For our problem, setting $S = T$ yields the 3-Free Set Problem
	\item<3-> Upper bounds for the 3-Free Set problem originated with Roth \cite{Roth53}
	\begin{itemize}
		\item Roth's Theorem shows that $|A| = o(n)$. In particular, Roth uses Fourier analysis to show $|A| = O(n / (\log \log n))$
		\item Best known upper bound: $O(n(\log \log n)^4 / \log n)$ and was shown by Bloom \cite{Bloom16}, using heavy machinery from additive combinatorics
	\end{itemize}
\end{itemize}
\end{frame}

\begin{frame}{Proof of a Sublinear Upper Bound}

\begin{itemize}
	\item Our goal is to show sublinearity of Our Combinatorial Problem
	\begin{itemize}
		\item The machinery in many of the proofs for 3-Free Sets do not easily generalize to analysis over 2 sets 
	\end{itemize}
\end{itemize}


\onslide<2->{\setbeamercolor{block title}{bg=Red, fg=white}
\begin{block}{Question}
	Is there a proof of sublinearity of 3-Free Sets that can generalize to our problem?
\end{block}
}

\onslide<3->{\setbeamercolor{block title}{bg=Green, fg=white}
\begin{block}{Answer}
	Yes! We can use a Graph Theoretic reduction and use Szemer\'{e}di's Regularity Lemma
\end{block}
}

\end{frame}


