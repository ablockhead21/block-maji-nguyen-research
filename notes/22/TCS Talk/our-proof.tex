\section{Proof of Our Result}
\begin{frame}{Graph Removal Lemma}

As a corollary of the Regularity Lemma, we have the following powerful result
\begin{lemma}[Graph Removal Lemma \cite{..}]
	Let $G$ be an $n$-vertex graph and let $H$ be a fixed $h$-vertex graph. If $G$ contains $o(n^h)$ copies of $H$, then $G$ can be made $H$-free by removing $o(n^2)$ edges.
\end{lemma}

\begin{itemize}
	\item By $H$-free, we mean that $G$ does not contain any copies of $H$ as a subgraph
	\item The Graph Removal Lemma can be restated more formally as follows
\end{itemize}

\end{frame}

\begin{frame}{Graph Removal Lemma}
\begin{lemma}[Graph Removal Lemma \cite{..}]
	$\forall \epsilon > 0$ and graphs $H$ on $h$ vertices
	\begin{description}[abc]
		\item[$\bullet$] $\exists \delta > 0$ such that
		\begin{description}[abc]
			\item[$\bullet$] $\forall G = (V,E)$ on $n$ vertices
			\begin{description}[abc]
				\item[$\bullet$] if $G$ contains at most $\delta n^h$ copies of $H$ $\epsilon n^2$-far from being $H$-free
				\begin{description}[abc]
					\item[$\bullet$] then $G$ can be made $H$-free by removing $\epsilon n^2$ edges
				\end{description}
			\end{description}
		\end{description}
	\end{description}
\end{lemma}

\begin{itemize}
	\item We use this version of the lemma to obtain our bound on $m$
\end{itemize}

\end{frame}

\begin{frame}{Proof Outline}
A high-level outline of the proof
\begin{enumerate}
	\item Transform our problem into a $3$-partite graph $G$
	\item Create the edge-set such that $G$ is a union of edge-disjoint triangles
	\item Use the Graph Removal Lemma to remove all triangle from $G$
	\item The number of edges removed exactly yields a bound on the size of our problem
\end{enumerate}
\end{frame}

\begin{frame}{Proof}
Suppose we have two sets $S$ and $T$ which satisfy Our Combinatorial Problem:
{\setbeamercolor{block title}{bg=ForestGreen, fg=white}
	\begin{block}{Our Problem}
		Find two ordered sets $S = \left\lbrace s_1,\dotsc, s_m\right\rbrace$ and $T = \left(t_1,\dotsc, t_m\right)$ such that
		\begin{itemize}
			\item $s_i$ and $t_i$ are non-negative integers for every $i$
			\item $s_i + t_j < n$ for every $i,j$
			\item $s_i + t_j \neq s_k + t_k$ for every $i,j,k$ that are not all identical
			\item $m$ is maximized
		\end{itemize}
\end{block}}

\begin{block}{Notation}
	For integer $n$, let $[n]\defeq \{1,2,\dotsc, n\}$
\end{block}

\begin{itemize}
	\item Construct a 3-partite graph $G$ with vertex sets $X = [n]$, $Y = [2n]$, and $Z = [3n]$
	\item Construct the edge sets as the union of the following sets
\end{itemize}

\end{frame}

\begin{frame}{Proof}

\begin{figure}[htp]
\begin{center}

\begin{adjustbox}{max width={\textwidth}, max height={.8\textheight}}

\begin{tikzpicture}

\node at (0,0) (X) {$X$};

\node at ($(X)+(0,-1)$) (x1) {$1$};
\node at ($(x1)+(0,-1)$) (x2) {$2$};
\node at ($(x2)+(0,-1)$) (xd1) {$\vdots$};
\node at ($(xd1)+(0,-1)$) (xn) {$n$};

\draw ($(x1)!0.5!(xn)$) ellipse (.5cm and 2cm);

\node at ($(X) + (.4\textwidth,0)$) (Y) {$Y$};

\node at ($(Y)+(0,-1)$) (y1) {$1$};
\node at ($(y1)+(0,-1)$) (y2) {$2$};
\node at ($(y2)+(0,-1)$) (yd1) {$\vdots$};
\node at ($(yd1)+(0,-1)$) (yn) {$n$};
\node at ($(yn)+(0,-1)$) (yd2) {$\vdots$};
\node at ($(yd2)+(0,-1)$) (y2n) {$2n$};

\draw ($(y1)!0.5!(y2n)$) ellipse (.6cm and 3cm);

\node at ($(Y)+ (.4\textwidth,0)$) (Z) {$Z$};

\node at ($(Z)+(0,-1)$) (z1) {$1$};
\node at ($(z1)+(0,-1)$) (z2) {$2$};
\node at ($(z2)+(0,-1)$) (zd1) {$\vdots$};
\node at ($(zd1)+(0,-1)$) (zn) {$n$};
\node at ($(zn)+(0,-1)$) (zd2) {$\vdots$};
\node at ($(zd2)+(0,-1)$) (z2n) {$2n$};
\node at ($(z2n)+(0,-1)$) (zd3) {$\vdots$};
\node at ($(zd3)+(0,-1)$) (z3n) {$3n$};

\draw ($(z1)!0.5!(z3n)$) ellipse (.7cm and 4cm);

\end{tikzpicture}	
	
\end{adjustbox}

\only<2>{$E(X,Y) = \big\lbrace (x, x+s_i)\;\colon\; x\in [n], i \in [m], s_i \in S \big\rbrace$}
\only<3>{$E(Y,Z) = \big\lbrace (x+s_i, x+s_i+t_i)\;\colon\; x\in[n], i\in[m], s_i \in S, t_i \in T \big\rbrace$}
\only<4>{$E(X,Z) = \big\lbrace (x, x+s_i+t_i)\;\colon\; x\in[n], i\in[m], s_i \in S, t_i \in T \big\rbrace$}

\end{center}
\end{figure}
	
\end{frame}

\begin{frame}{Proof}

\begin{block}{Edge Set of $G$}
$E(G)$ is the union of
\begin{itemize}
	\item $E(X,Y) = \big\lbrace (x, x+s_i)\;\colon\; x\in [n], i \in [m], s_i \in S \big\rbrace$
	\item $E(Y,Z) = \big\lbrace (x+s_i, x+s_i+t_i)\;\colon\; x\in[n], i\in[m], s_i \in S, t_i \in T \big\rbrace$
	\item $E(X,Z) = \big\lbrace (x, x+s_i+t_i)\;\colon\; x\in[n], i\in[m], s_i \in S, t_i \in T \big\rbrace$
\end{itemize}
\end{block}
\begin{itemize}
	\item Since $x \in [n]$ and $|S|=|T|=m$, we have
	\begin{align*}
	|E(X,Y)| = &|E(Y,Z)| = |E(X,Z)| = nm\\
	\implies &|E(G)| \leq 3mn
	\end{align*}
	
	\item Next, we claim the following about the constructed $E(G)$:
\end{itemize}

\end{frame}

\begin{frame}{Proof}

\begin{claim}
	$E(G)$ is a union of edge-disjoint triangles.
\end{claim}

\begin{block}{Note}
	A triangle is defined by the tuple of vertices $(x,y,z)$ such that $\big\lbrace (x,y), (y,z), (x,z) \big\rbrace$ are edges in $G$
\end{block}

\begin{itemize}
	\item For all $x\in [n]$, $i\in [m]$, $s_i \in S$, and $t_i \in T$, the tuple $(x,\; x+s_i,\; x+s_i+t_i)$ is a triangle in $G$.
	
	\item This follows by construction of $E(G)$ since
	\begin{itemize}
		\item $E(X,Y) = \big\lbrace (x, x+s_i)\;\colon\; x\in [n], i \in [m], s_i \in S \big\rbrace$
		\item $E(Y,Z) = \big\lbrace (x+s_i, x+s_i+t_i)\;\colon\; x\in[n], i\in[m], s_i \in S, t_i \in T \big\rbrace$
		\item $E(X,Z) = \big\lbrace (x, x+s_i+t_i)\;\colon\; x\in[n], i\in[m], s_i \in S, t_i \in T \big\rbrace$
	\end{itemize}
	\item We proceed by contradiction
\end{itemize}
\end{frame}

\begin{frame}{Proof}
\begin{itemize}
	\item Suppose there exist two non-edge-disjoint triangles $(x,\; x+s_i,\; x+s_i+t_i)$ and $(y,\; y+s_j,\; y+s_j + t_j)$ in $G$
	\item Let $(x,x+s_i) = (y,y+s_j)$ be the shared edge
	\begin{itemize}
		\item This implies $x = y$ and $s_i = s_j$
		\item Since $S$ is a solution to our problem, $s_i = s_j$ if and only if $i = j$.
		\item Since $i = j$, we have that $t_i = t_j$ also; so the two triangles are the same (contradiction)
	\end{itemize}
\end{itemize}
\end{frame}

\begin{frame}{Proof}
\begin{itemize}
	\item Suppose there exist two non-edge-disjoint triangles $(x,\; x+s_i,\; x+s_i+t_i)$ and $(y,\; y+s_j,\; y+s_j + t_j)$ in $G$
	\item Let $(x+s_i,\; x+s_i+t_i) = (y + s_j,\; y+s_j+t_j)$ be the shared edge
	\begin{itemize}
		\item This implies that
		\begin{align}
			x+s_i &= y + s_j\\
			x+s_i+t_i &= y + s_j + t_j
		\end{align}
		\item Substituting equation (1) into (2) we have
		\begin{align*}
			y+s_j+t_i = y + s_j + t_j
		\end{align*}
		\item This implies $t_i = t_j$, which again implies $i = j$
		\item So $s_i = s_j$ and by equation (1), $x = y$
		\item Again, these triangles are the same (contradiction)
	\end{itemize}
\end{itemize}
\end{frame}

\begin{frame}{Proof}
\begin{itemize}
	\item Suppose there exist two non-edge-disjoint triangles $(x,\; x+s_i,\; x+s_i+t_i)$ and $(y,\; y+s_j,\; y+s_j + t_j)$ in $G$
	\item Let $(x,\; x+s_i+t_i) = (y,\; y + s_j + t_j)$ be the shared edge
	\begin{itemize}
		\item This implies $x = y$ and $s_i + t_i = s_j + t_j$
		\item If $i \neq j$, then $(s_i, t_i)$ and $(s_j, t_j)$ are a witness to $S$ and $T$ not satisfying our problem (contradicts our assumption about $S$ and $T$)
		\item It must be the case $i = j$, and again these two triangles are the same (contradiction)
	\end{itemize}
	\item Therefore, all triangles are edge-disjoint!
\end{itemize}
\end{frame}

\begin{frame}{Proof}
\begin{itemize}
	\item Since $G$ is a union of edge-disjoint triangles, we have $E(G) = 3mn$
	\item Further, the number of triangles in $G$ is exactly $nm$
	\item Since $S, T \subseteq [n]$, we have that
	\begin{align*}
	nm \leq n^2 = o(n^3)
	\end{align*}
	\item For sufficiently large $n$ we know $n^2 < \delta n^3$, where $\delta$ is defined in the Graph Removal Lemma
	\item So $G$ has at most $\delta n^3$ triangles
\end{itemize}
\end{frame}

\begin{frame}{Proof}
\begin{itemize}
	\item Thus by the Graph Removal Lemma, we can remove $\epsilon n^2$ edges to make $G$ triangle-free
	\item Since there are $nm$ edge-disjoint triangles in $G$, it is sufficient to remove one edge from each triangle to destroy all triangles in $G$
	\begin{itemize}
		\item So we can remove $nm$ edges to destroy all triangles
		\item This implies $nm = \epsilon n^2$, and thus $m = \epsilon n$
	\end{itemize}
\end{itemize}
\end{frame}

\begin{frame}{Proof}
\begin{itemize}
	\item As noted in \cite{..}, $\delta^{-1}$ is a tower of twos of height proportional to $\epsilon^{-5}$
	\begin{itemize}
		\item Choose $n$ large enough such that $\log^* n = \epsilon^{-5}$
		\item This implies 
		\begin{align*}
			\delta^{-1} &= 2^{2^{\iddots^2}} \bigg\rbrace O(\epsilon^{-5})\\
			&= 2^{2^{\iddots^2}} \bigg\rbrace O(\log^* n)\\
			&= O(n)
		\end{align*}
		\item This implies that
		\begin{align*}
			\epsilon \delta^{-1} = O(\epsilon n) = O\left(\frac{n}{\log^* n}\right) < n
		\end{align*}
	\end{itemize}
\end{itemize}
\end{frame}

\begin{frame}{Proof}
\begin{itemize}
	\item In particular, $\epsilon n = O\left(\frac{n}{\log^* n}\right)$
	\item This implies
	\begin{align*}
		m = \epsilon n = O\left(\frac{n}{\log^* n}\right)
	\end{align*}
\end{itemize}
\end{frame}



