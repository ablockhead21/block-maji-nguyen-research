\section{Connection to the Tri-Colored Sum-Free Set Problem}
\begin{frame}{Connection to Tri-Colored Sum-Free Set Problem}

We saw before that our problem is a more general version of the 3-Free Set problem	

\onslide<2->{\setbeamercolor{block title}{bg=Red, fg=white}
\begin{block}{Question}
	Is our problem related to other problems in additive combinatorics?
\end{block}
}

\onslide<3->{
{\setbeamercolor{block title}{bg=Green, fg=white}
\begin{block}{Answer}
	Yes, The Tri-Colored Sum-Free Set Problem!
\end{block}
}}

\onslide<4->{\setbeamercolor{block title}{bg=CornflowerBlue, fg=white}
\begin{definition}[Tri-Colored Sum-Free Set (TCSF) \cite{BCCGNSU17}]
	For any abelian group $G$, a {\em tri-colored sum-free set} in $G$ is a collection of triples $\{(a_i, b_j, c_k)\}$ in $G$ such that $a_i + b_j + c_k = 0$ if and only if $i = j = k$.
\end{definition}
}
\begin{itemize}
\item<5-> The TCSF Set problem asks what is the largest TCSF Set which can occur in $G$
\begin{itemize}
	\item Problem is also knows as the {\em largest cap set} problem
\end{itemize}

\end{itemize}
\end{frame}

\begin{frame}{Connection to Tri-Colored Sum-Free Set Problem}
\begin{itemize}
	
	\item Notice for $G = \bbZ$, the set $\{(x, x, -2x)\}$ for $x \in \bbZ$ is exactly the 3-Free Set problem
	
	\item<2> In fact for $G= \bbZ$, the TCSF Set
	\begin{align*}
		\{ (s_i, t_j, -(s_k+t_k)) \}
	\end{align*} 
	is identical to our problem!
\end{itemize}
\end{frame}

\begin{frame}{Tri-Colored Sum-Free Set Problem}
\begin{itemize}
	\item Best-known upper bound results about the TCSF were proven only recently (May 2016)
	{\setbeamercolor{block title}{bg=Coral, fg=white}
	\begin{theorem}[\cite{BCCGNSU17}, Theorem 4.14]
		If $q$ is a prime power and $C_q$ is the cyclic group of order $q$, then sum-free sets in $C_q^n$ have size at most $3\theta^n$, where
		\begin{align*}
			\theta = \underset{\rho > 0}{\min}\; (1 + \rho + \cdots + \rho^{q-1})\rho^{-(q-1)/3}
		\end{align*}
	\end{theorem}}
\end{itemize}
\end{frame}

\begin{frame}{Tri-Colored Sum-Free Set Problem}
\begin{itemize}
	\item More recently (June 2016), this result was shown to be tight within a subexponential factor
	{\setbeamercolor{block title}{bg=Coral, fg=white}
	\begin{theorem}[\cite{KSS16}, Theorem 2]
		Fix an integer $q \geq 2$ and define $\theta$ as above. For $n$ sufficiently large, there are sum-free sets in $C_q^n$ with size at least
		\begin{align*}
		\theta^n \exp\left( -2 \sqrt{2(\log 2)(\log \theta)n} - O_q(\log n) \right)
		\end{align*}
	\end{theorem}}
	where $O_q(\cdot)$ refers to bounds as $n \rightarrow \infty$ through integers divisible by $3$ and $q$ fixed
\end{itemize}
\end{frame}

\begin{frame}{Final Remarks}
\begin{itemize}
	\item<1-> Upper bounds beating $O(n / \log^* n)$ for $G = \bbZ$ in the TCSF Set Problem have not been shown yet
	\item<2-> This is an active problem in additive combinatorics with many applications
	\begin{itemize}
		\item<3-> Matrix multiplication \cite{KSS16}
		\item<3-> Property testing \cite{BX15}
		\item<3-> Removal lemmas in additive combinatorics \cite{Green05}
	\end{itemize}
\end{itemize}
\end{frame}

\begin{frame}{Thank You!}
\centering

Any questions?
\end{frame}




