\section{Szemer\'{e}di's Regularity Lemma}
\begin{frame}{Szemer\'{e}di's Regularity Lemma}
	
We'll carefully define all the necessary components needed for Szemer\'{e}di's Regularity Lemma.
\begin{itemize}
	\item<2-> Let $G = (V,E)$ be a simple, undirected graph on $n$ vertices
	\item<3-> If $X \subseteq V$ and $Y \subseteq V$, we define $e(X,Y)$ as the number of edges between $X$ and $Y$
\end{itemize}
\onslide<4->{\setbeamercolor{block title}{bg=CornflowerBlue, fg=white}
\begin{definition}[Edge Density]
	For disjoint $X,Y \subseteq V$, we define the {\em edge density} as
	\begin{align*}
		d(X,Y) = \dfrac{e(X,Y)}{|X|\cdot |Y|}
	\end{align*}
\end{definition}}
	
\end{frame}

\begin{frame}{Szemer\'{e}di's Regularity Lemma}
Next we have the Regularity Condition
{\setbeamercolor{block title}{bg=CornflowerBlue, fg=white}
\begin{definition}[$\eps$-Regular Pairs]
	Let $\eps > 0$. Given disjoint $X, Y \subseteq V$, the pair $(X,Y)$ is {\em $\epsilon$-regular} if for every $A \subseteq X$ and $B \subseteq Y$ such that
	\begin{align*}
		|A| > \epsilon|X| & & \text{and} & & |B| > \epsilon|Y|
	\end{align*}
	we have
	\begin{align*}
		|d(A,B) - d(X,Y)| < \epsilon
	\end{align*}
\end{definition}}
\end{frame}

\begin{frame}{Szemer\'{e}di's Regularity Lemma}

{\setbeamercolor{block title}{bg=Coral, fg=white}
\begin{theorem}[Regularity Lemma \cite{Szemeredi75,KSSS02}]
\onslide<1->{$\forall \eps > 0$}
\begin{description}[abc]
	\item<2->[$\bullet$] $\exists M = M(\eps) \in \bbZ$ such that
	\begin{description}[abc]
		\item<3->[$\bullet$]  $\forall$ simple undirected  $G=(V,E)$ on $n$ vertices
		\begin{description}[abc]
			\item<4->[$\bullet$]  $\exists$ a partition of $V$ into $k$ classes $V_1, V_2, \dotsc, V_k$ such that
			\begin{description}[abc]
				\item<5->[$\bullet$] $k \leq M$
				\item<6->[$\bullet$] $|V_i| \leq \ceil{\epsilon |V_j|}$ for every $i$
				\item<7->[$\bullet$] $\abs{ |V_i| - |V_j| } \leq 1$ for all $i,j$ (equipartition)
				\item<8->[$\bullet$] $(V_i, V_j)$ is $\eps$-regular in $G$ for all except at most $\eps k^2$ pairs $(i,j)$
			\end{description}
		\end{description}
	\end{description}
\end{description}
\end{theorem}}

\end{frame}

\begin{frame}{Szemer\'{e}di's Regularity Lemma}
The Regularity Lemma is a very powerful tool in Extremal Graph Theory
 
\begin{block}{Intuitive Statement}
	\ul{Any large and dense enough} graph ``looks like'' a union of a small amount of \ul{random-looking} bipartite graphs.
\end{block}

\begin{itemize}
	\item<2-> Large enough graphs can be approximated by a small number of uniform bipartite graph
	\item<3-> For this reason, the Regularity Lemma is also referred to as the Uniformity Lemma
\end{itemize}

\onslide<4->{
\begin{block}{Note}
	Gowers \cite{Gowers1997} showed that the bound obtained for $M(\eps)$ is a tower of 2's of height proportional to $\eps^{-5}$.
	\begin{itemize}
		\item He showed this is an \ul{inherent feature} of the Regularity Lemma
	\end{itemize}
\end{block}}

\end{frame}

\begin{frame}{Szemer\'{e}di's Regularity Lemma}
Szemer\'{e}di created this lemma to prove his famous result

{\setbeamercolor{block title}{bg=Coral, fg=white}
\begin{theorem}[Szemer\'{e}di's Theorem \cite{Szemeredi75-2}]
$\forall k \geq 3$ and $\delta > 0$
\begin{description}[abc]
	\item<2->[$\bullet$] $\exists n_0 \in \bbZ$ such that
	\begin{description}[abc]
		\item<3->[$\bullet$] $\forall N \geq n_0$
		\begin{description}[abc]
			\item<4->[$\bullet$] If $A \subseteq \{1,2,\dotsc,N\}$ and $|A| \geq \delta N$
			\begin{description}[abc]
				\item<5->[$\bullet$] then $A$ contains an arithmetic progression of length $k$
			\end{description}
		\end{description}
	\end{description}
\end{description}
\end{theorem}}

\begin{itemize}
	\item<6-> For $k = 3$, we have exactly Roth's Theorem for 3-Free Sets
	\begin{itemize}
		\item We follow the proof of Roth's Theorem using the Regularity Lemma to prove our result
	\end{itemize}
\end{itemize}
\end{frame}


