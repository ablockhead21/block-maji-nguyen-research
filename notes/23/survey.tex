\section{Survey}
Here, we have a survey of results and open problems concerning LPN.

\begin{definition}[(Search) Learning Parity with Noise~\cite{??}]
	For $\tau \in (0,1/2)$, $\ell \in \bbN$, the search $\pred{LPN}_{\tau, \ell}$ problem is $(n, t, \epsilon)$-hard if for every distinguisher $\pred{D}$ running in time $t$,
	\begin{align*}
		\underset{\vec{s}, \vec{A}, \vec{e}}{\pr}\left[\pred{D}\left(\vec{A}, \vec{As}\xor\vec{e}\right)=\vec{s}\right] \leq \epsilon,
	\end{align*}
	where $\vec{s} \getsr \bbZ^\ell_2$, $\vec{A} \getsr \bbZ^{n\times \ell}_2$, and $\vec{e} \drawn \pred{Ber}^n_\tau$.
\end{definition}
Next we have the seemingly stronger version of LPN, namely Decisional LPN.
\begin{definition}[(Decisional) Learning Parity with Noise~\cite{??}]
	For $\tau \in (0,1/2)$, $\ell \in \bbN$, the search $\pred{LPN}_{\tau, \ell}$ problem is $(n, t, \epsilon)$-hard if for every distinguisher $\pred{D}$ running in time $t$,
	\begin{align*}
	\abs{\underset{\vec{s}, \vec{A}, \vec{e}}{\pr}\left[\pred{D}\left(\vec{A}, \vec{As}\xor\vec{e}\right)\right] -
	\underset{\vec{r}, \vec{A}}{\pr}\left[\pred{D}\left(\vec{A}, \vec{r} \right)=\vec{s}\right] } \leq \epsilon,
	\end{align*}
	where $\vec{s} \getsr \bbZ^\ell_2$, $\vec{A} \getsr \bbZ^{n\times \ell}_2$, $\vec{e} \drawn \pred{Ber}^n_\tau$, and $\vec{r}\getsr\bbZ^n_2$.
\end{definition}

\noindent Here is a summary for LPN. 
\begin{enumerate}
	\item Decisional and search problems are polynomially equivalent \cite{C:BFKL93,JC:KatShiSmi10}.
	The following lemma gives the result formally.
	
	\begin{importedlemma}[{\cite[Lemma 1]{JC:KatShiSmi10}}]
		If decisional $\LPN_{\tau, \ell}$ is not $(q,t,\eps)$ secure, then search $\LPN_{\tau,\ell}$ is not $O(q',t',\eps')$ secure, where
		\begin{align*}
			q' = O(q\cdot \log \ell/\eps^2) & & t' = O(t\cdot \ell \cdot \log \ell/\eps^2) & & \eps' = \eps/4.
		\end{align*}
	\end{importedlemma}
	
	\item The search LPN problem can be stated as the NP-complete problem of decoding random linear code \cite{BerlMcElTilb78}.
	\item The best known algorithms to recover an $ \ell $ bit secret need $ 2^{\Theta(\ell / \log \ell)} $ time and samples \cite{STOC:BluKalWas00,Levieil06}. 
	If given only polynomially many $ q = \pred{poly}(\ell) $ samples, the running time of best known algorithms becomes to $ 2^{\Theta(\ell/ \log \log \ell)} $ \cite{Lyubashevsky05}.
	Further, when given only linearly many samples $ q = \Theta(\ell) $, the best known algorithms have running time $ 2^{\Theta(\ell)} $ \cite{AC:MayMeuTho11,Jacques89}. 
	Unlike most others, no quantum algorithms for LPN are known which are significantly faster than the classical ones. 
	
	\item The secret $ s $ is usually assumed to be uniformly random. 
	This is one of the hardest distributions.  
	The LPN problem where $ \vec{s} \getsr \pred{Ber}^n_\tau $ is as hard as for uniform $ \vec{s} \getsr  \bbZ^n_2 $ \cite{C:ACPS09}. 
	\textcolor{red}{Nothing is known about the hardness of LPN for \textit{general} distributions of high min-entropy.} 
	However, subspace LPN problem is equivalent to (as hard as) the LPN problem. 
	In this case, the secret $s \in \bbZ^\ell_2$ is uniformly sampled from $V \subseteq \bbZ^\ell_2$ such that $\dim(V) = \ell' \leq \ell$ (\ie, any $s \getsr V$ has min-entropy $\ell'$).
	This can be shown to be exactly as hard as the standard LPN problem with an $\ell'$-bit uniform e
	
	\item Sampling of the random matrix $ A  \getsr \bbZ^{n\times \ell}_2 $ is the most expensive part in generating LPN. Using Toeplitz matrix requires only $ n + \ell $ random bits (as opposed $ n \cdot \ell $) \cite{??}. \textcolor{red}{Can we use  Wozencraft's technique to save randomness?} One major drawback of this is that it requires $ n = 2 \ell $.
	\item  Exact LPN: a minor variation of LPN where $ \vec{e} $ has weight exactly $ n \tau $. \textcolor{red}{It is open whether decisional XLPN is equivalent to LNP or not.} 
\end{enumerate} 

\noindent Constructions of crypto primitives from LPN. 
\begin{enumerate}
	\item LPN -> OWF
	\item LPN -> PRG
	\item LPN -> secret-key encryption \cite{??}
	\item LPN -> PKE \cite{??} (CRYPTO 2016). \textcolor{red}{More efficient PKE form constant-noise LPN?}
	\item LPN -> PRF \cite{??} (EUROPCRYPT 2016). \textcolor{red}{Constant depth construction?}
	\item \textcolor{red}{LPN -> CRHF}
	\item \textcolor{red}{LPN -> FHE}
\end{enumerate}