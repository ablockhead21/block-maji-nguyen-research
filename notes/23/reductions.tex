\subsection{\LPN Reductions}
In this section we attempt to show some reductions based on the \LPN assumption.
In particular, we are interested in proving either
\begin{itemize}
	\item \LPN is still hard when using a Toeplitz matrix for $\bA$ instead of a uniformly random matrix $\bA$.
	\item \LPN is still hard when using a secret $\bs$ drawn from a general min-entropy source.
\end{itemize}
Let \TLPN denote the \LPN problem where the matrix $\bA$ is a Toeplitz matrix.
We wish to show the following result.
\begin{claim}
	If $\LPN_{\tau,k}$ is $(n,t,\eps)$-hard, then $\TLPN_{\tau,k}$ is $(n',t',\eps')$-hard for $n' = f(n,t,\eps)$, $t' = g(n,t,\eps)$, and $\eps' = h(\eps)$, where $f,g,h$ are some functions.
\end{claim}
Towards this direction, we can choose to show decisional \LPN implies decisional \TLPN, or search \LPN implies search \TLPN.
It is unknown if search \TLPN is equivalent to decisional \TLPN, though it certainly seems to be the case.
We attempt to prove the result, following the proof of equivalence of search and decisional \LPN given in \importedlemmaref{lpn-poly} (\cite[Lemma 1]{JC:KatShiSmi10}).
We explicitly state the lemma as it is presented in \cite{JC:KatShiSmi10}.
\begin{importedlemma}
	Suppose there exists an algorithm $D$ making $n$ oracle queries, running in time $t$, and with
	\begin{align*}
		\abs{ \Pr\left[\bs \getsr \zo^k~\colon~ D^{A_{\bs,\tau}}(1^k) = 1\right] - \Pr\left[D^{U_{k+1}}(1^k)=1\right] } \geq \eps.
	\end{align*}
	Then there exists an algorithm $M$ making $n' = O(n\eps^{-2}\log k)$ oracle queries, running in time $t' = O(tk\eps^{-2}\log k)$ such that
	\begin{align*}
		\Pr[\bs \getsr \zo^k~\colon~ M^{A_{\bs,\tau}}(1^k)=\bs] \geq \eps/4
	\end{align*}
\end{importedlemma}
\noindent Here, $A_{\bs,\tau}$ is an oracle which outputs independent samples according to the distribution
\begin{align*}
	\left\{ \ba\getsr \zo^k; e \getsr \Ber_\tau\colon (\ba,\ip{\ba}{\bs}\oplus e) \right\}.
\end{align*}
This immediately highlights the main technical difficulty in proving that search and decisional \TLPN are (polynomially) equivalent.
Namely, the distribution/oracle does not return independent samples, save for the first sample.
In particular, let $T_{\bs, \tau}$ be the oracle which samples according to the following distribution: define $T_{\bs, \tau}(i)$ as follows.
If $i = 1$, then output
\begin{align*}
	(\ba_1, \ip{\ba_1}{\bs}\oplus e_1)
\end{align*}
where $\ba_1 \getsr \zo^k$ and $e_1 \getsr \Ber_\tau$.
Else, for all $i \geq 2$, output
\begin{align*}
	(\ba_i, \ip{\ba_i}{\bs}\oplus e_i)
\end{align*}
where $e_i \getsr \Ber_\tau$ and $\ba_i \defeq (b\Vert(\ba_{i-1})\sub{k-1})$ where $b \getsr \zo$.
In other words, the first bit of $\ba_i$ is chosen uniformly at random and the remaining $k-1$ bits are the first $k-1$ bits of $\ba_{i-1}$.
Note that each $e_i$ is sampled independently, each $b$ is sampled independently, and that $i$ represents the $i^{th}$ access to the oracle $T_{\bs,\tau}$.
It is easy to see the output distribution of accessing the oracle $T_{\bs,\tau}$ $n$ times is identical to the product $\bA\bs$, where $\bA$ is an $n\times k$ Toeplitz matrix.


